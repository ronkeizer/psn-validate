\documentclass[a4,11pt]{report} \usepackage[pdftex]{graphicx}
\usepackage{setspace}
%\usepackage{lineno}
\usepackage{color}
\usepackage{float}
\usepackage{enumitem}
\usepackage{xspace}
\usepackage{lscape}
\usepackage{listings}
\usepackage{bbm}
\definecolor{PiranaOrange}{rgb}{0.8,0.3,0.08}
\definecolor{Red}{rgb}{0.4,0.0,0.0}
\definecolor{Green}{rgb}{0.0,0.4,0.0}
\definecolor{Blue}{rgb}{0.0,0.0,0.4}
\definecolor{Grey}{rgb}{0.4,0.4,0.4}
\definecolor{LightGrey}{rgb}{0.92,0.92,0.92}

\bibliographystyle{unsrt}%Choose a bibliograhpic style}
%\usepackage{utopia} %\usepackage{charter} %\usepackage{palatino}
%\usepackage{bookman} %\usepackage{newcent} %\usepackage{times}
%\usepackage[options]{natbib} \sloppy
\renewcommand{\familydefault}{\sfdefault}
\oddsidemargin 1.5cm
\textwidth 14cm

\newcommand{\test}{\textcolor{Blue}{\textit{test}}\xspace}
\newcommand{\reference}{\textcolor{Green}{\textit{reference}}\xspace}
\newcommand{\valpsn}{\textcolor{Blue}{\textit{valpsn}}\xspace}
\newcommand{\ValPsN}{\textcolor{PiranaOrange}{\textit{ValPsN}}\xspace}
\newcommand{\psn}[1]{\textcolor{Grey}{#1}}

%% code listings
\lstset{
  language=fortran,                   % the language of the code
  basicstyle=\ttfamily\scriptsize,    % the size of the fonts that are used for the code
  commentstyle=\color{Grey}\textit,
  backgroundcolor=\color{LightGrey},  % choose the background color. You must add \usepackage{color}
  captionpos=b,                       % sets the caption-position to bottom
  columns=flexible,
  showspaces=false,
  showstringspaces=false,
  showtabs=false,
  morekeywords={*, ...}               % if you want to add more keywords to the set
}

\begin{document}

\title{
  \vspace{-100pt}
  \textbf{
  \textcolor{Blue}{\Large PsN Validation Tool}
  }\\
  \vspace{5pt}
  \scriptsize \textcolor{Grey}{Pirana Software \& Consulting BV} \\
  \normalsize
  \vspace{15pt}
  \hspace{15pt}\includegraphics[scale=0.15]{images/pirana_logo_blue.jpg}\\
  \vspace{15pt}
  \scriptsize Version 1.0.0\\Installation guide and Manual \\
  \vspace{5pt}
  \scriptsize {\today} \\
  \date{}
}
\maketitle

\tableofcontents

\chapter{Introduction}

\ValPsN is a software tool developed by Pirana Software \& Consulting
BV for comparing and validating output of PsN and NONMEM. \ValPsN
compares the numeric output from a sequence of PsN runs (execute,
bootstrap, vpc, etc...) with that of reference PsN runs. Differences in
PsN output might result either from differences in NONMEM output, or
differences in subsequent numerical operations performed by
PsN. Therefore, by `validating' PsN, at the same time the
underlying NONMEM infrastructure is tested and validated.

\vspace{10pt}

\noindent This installation guide and manual contains important
guidance for the installation and use of \ValPsN, and should be read
before setting up a validation. We welcome suggestions for further
improvements and bug reports.

\vspace{10pt}

\section{How it works}
The basic principle is simple: \ValPsN compares output produced from a
specific installation of PsN on your system (\test), with output
derived from an earlier PsN run (\reference). It is thus implicitly
assumed that the output from the \reference is trustworthy and
correct. While the framework for the validation is implemented in
Perl, the comparison of the numeric output is performed in R, using
predefined R-scripts. It is however also possible to
specify a custom R-script if it adheres to a few requirements
(specified in the Reference chapter).

\vspace{10pt}

\noindent The proposed workflow for a validation of PsN is as follows:
\begin{enumerate}
\item Collect a reference library with output from PsN toolkit that is
  known / believed to be correct. Of course, for an appropriate
  validation of PsN, output from all toolkit commands available in PsN, or at
  least from a selection of the most often used tools should be included in
  the validation run.
\item Create a configuration file for \ValPsN. This is a flat text file
  that specifies e.g. which PsN version to use, where to store the
  test output, which PsN tools to run, as well as the speficic options for
  each tool. Please see the Reference chapter of this manual.
\item Run \ValPsN and generate the \test output, which is then
  compared to the \reference output. You can also choose to use
  previously generated output as \test output, instead of rerunning
  the specific PsN tools in the validation run itself.
\end{enumerate}

\section{Disclaimers}
\begin{description}
\item[PsN] You will note that the command line interface and the
  syntax of the configuration file of ValPsN closely resemble those in
  PsN. This is to facilitate the use for those already accustomed
  to PsN. However, we would like to stress that ValPsN does not use
  any PsN library nor does it share any code with PsN.

\item[Operating system] The current version of ValPsN is developed
  primarily for Linux and Mac OSX. Although this has not been tested
  yet, the software should run fine on Windows as well, since only
  standard Perl and PsN toolkit commands are used. This installation
  guide and manual assumes Unix-type OS.

\item[Sequential execution] The \ValPsN script is implemented
  sequentially: it will run a sequence of PsN commands (each one
  followed by execution of an R script), in the order specified in the
  configuration file. At current, there is no option to run these
  tests in parallel, which implies that a full validation procedure
  may take a long time. An easy workaround for this is to split up the
  validation procedure in separate validations that can be run in
  parallel, e.g. a validation run in which a set of \psn{execute}
  commands is performed, one run in which a \psn{bootstrap} is run,
  and one in which the other PsN tools such as \psn{vpc}, \psn{llp},
  \psn{scm} are validated. Of course the {\tt -threads} option in PsN
  can be used to parallelize single validation steps.

\end{description}

\chapter{Installation}

A zip-file containing the \ValPsN software is provided
(e.g. valpsn\_[version].zip), either from the Pirana website or from a
different download site provided by the developer. Unpack the zip
file, and copy the contents to an appropriate folder on the machine or
server of interest, e.g. in {\tt /opt/valpsn/}, or {\tt
  /home/username/valpsn}.

\vspace{10pt}
\noindent Check that in your installation folder you now have at
least the following files and folders:

\begin{lstlisting}
/doc           # Contains some documents, e.g. this manual
/ini           # Example setup up files for validation runs
/library       # Library of example PsN output files
/R             # Default validation R scripts
valpsn.pl      # the ValPsN tool
\end{lstlisting}

\noindent For ValPsN to function properly, three things need to be accessible on
the machine or on the cluster.
\begin{itemize}
\item one (or more) installations of PsN
\item one (or more) installations of NONMEM
\item an installation of R (available from http://cran.r-project.org)
\end{itemize}

\noindent To test if ValPsN is set up correctly, and is able to run PsN tools
and invoke NONMEM, run this command (from the folder where ValPsN is
installed):

\vspace{5pt}
\begin{lstlisting}
perl valpsn.pl ini/test.ini
\end{lstlisting}
\vspace{5pt}

\noindent This will run a quick validation run, and invokes PsN,
NONMEM and R using default settings. It will create the folder {\tt
  ~/valpsn\_test} in your home folder (unless otherwise specified in
{\tt test.ini}). Completion should take only a few seconds, and should
report that the validation succeeded. Of course, this is not part of a
validation yet, it just tests if the validation infrastructure is
installed and accessible by \ValPsN. If however one of these three
tools spawns an error, or the tools cannot be accessed, please fix
these respective problems before continuing validation. It is
difficult to give specific troubleshooting advice here, but problems
may be due to e.g. incorrectly set paths, unavailable or incorrectly
set environment variables, or insufficient access priviliges.

\chapter{Reference manual}

In the file {\tt val\_ex\_1.ini} an example validation is
implemented. Note that all files for this example, including model
files and datasets are included in the distribution, as well as the
output from the PsN commands, so this example can be run as a
validation in itself. Feel free to include these examples in your own
validation library as well.

\section{Setting up the validation}
\vspace{10pt} In the main folder, locate the file {\tt val\_ex\_1.ini},
and make a copy, e.g. to the file {\tt test1.ini}, and edit
to your liking. The contents of this configuration file are explained
step-by-step below.

\vspace{5pt}

\noindent The {\tt [folder]} section defines the folder structure,
e.g. where \ValPsN can find the model library, and where it should
put the output:

\begin{lstlisting}
[folders]
lib_folder      = /Users/ronkeizer/Dropbox/projects/psn_validate/library
			  # Absolute path to model library
out_folder      = /Users/ronkeizer/ValPsn/runs
			  # Absolute path where to run validation tests
R_script_folder = /Users/ronkeizer/Dropbox/projects/psn_validate/R
report_folder   = /Users/ronkeizer/ValPsn/reports
			  # Absolute path to folder for validation reports
run_folder      = PsN_3.5.3_20120930
                          # Folder for this specific validation run,
			  # will be created below the "out_folder"
report          = PsN_3.5.3_20120930.txt
                          # Report file to generate, will be
			  # created in the "report_folder"
\end{lstlisting}

\vspace{5pt}

\noindent Secondly, the software settings for the validation run are defined:

\begin{lstlisting}
[software]
perl_executable=perl      # location of perl executable, if not in PATH
psn_executables=	  # location of folder with PsN executables,
			  # No need to specify this if in PATH.
psn_version=              # Optionally specify a different version to
                          # use, e.g. psn_vesion=3.5.3. If not speci-
                          # fied, the default version is used.
R_executable=R            # Location of R executable, if not in PATH
nm_mod_extension=mod      # File extension of NONMEM model files
nm_out_extension=lst      # Filename extension of NONMEM output files
\end{lstlisting}

\vspace{5pt}

\noindent The next section defines the default settings for the
validation run. These settings will be carried over into each specific
validation step, unless overriden in the respective step itself.

\begin{lstlisting}
[validation]              
                          # General validation settings
                          # Note: all these settings can be overridden
			  # in each specific PsN test!
run_psn=1		  # Default is 1|TRUE. set this to FALSE if you
			  # only want to generate a validation report
			  # based on output from a previous validation
			  # procedure.
run_test=1		  # Actually perform the tests using the R-scripts?
			  # By default set to 1
verbose_level=2           # 1: condensed info
			  # 2: extended run/test info (default)
			  # 3: also include NONMEM output
nm_version=default        # NONMEM version to test. Can  also be
			  # issued specifically per test, which
			  # overrides the general setting. If not
			  # specified, PsN will use "default".
extra_arguments=	  # Optional arguments to add to all PsN
			  # commands, e.g.:
			  # -clean=1 -model_dirname
threads=5                 # Number of threads to use for PsN commands
tolerance=0.005  	  # Allowed relative difference between test
		 	  # and reference estimate. Specify as
		 	  # fraction: (test-ref)/ref. So the default
		 	  # 0.001 = 0.1\% difference allowed.
clean=2                   # PsN -clean option: (0/1/2/3)
\end{lstlisting}

\vspace{5pt}
\noindent The {\tt tolerance} parameter is important. It will be
obvious that too high a tolerance will render your validation
unuseful, but specifying a too low tolerance might fail your
validation on too many tests. Small changes in results can be expected
between NONMEM runs, and can be due to differences in (in order of
likeliness) the NONMEM version, Fortran compiler and compiler
settings, machine hardware and operating system. A tolerance of 0.005
is suggested for comparing the parameter estimates for the
\psn{execute} command, which will most likely be small enough to
detect relevant differences, but large enough to not be affected too
much by the issues mentioned above. Of course, if your desire is to
achieve perfect agreement between test and reference numbers, you can
always set the tolerance to a lower number. For tools like boostrap
and VPC, it is advised to set the tolerance (much) higher when the
{\tt -seed} function is not used.

\vspace{5pt}

\noindent What follows next, are the specification settings for each
validation step. There is no limit on the number of tests to be
included in the validation run, and of course multiple tests can (and
should) be run for the different PsN toolkit functions. Each
specification of a test starts with the header that specifies the function
in brackets, followed by the specific settings for that tool, e.g.

\begin{lstlisting}
[execute]
folder=NM_examples
model=run4.mod
reference=run4.lst
# tests
test_ofv=TRUE
test_parameters=TRUE
# /tests
\end{lstlisting}

\vspace{5pt}

\noindent Chapter 3.3 contains a reference manual for each specific PsN
tool.

\newpage

\section{Running the validation}

Every validation run is started by invoking the \psn{valpsn.pl}
script. This bash script invokes the main perl script that acutally
performs the validation. The script will read in the configuration
file, and perform all the specified validation elements in the
sequence specified in the configuration file, e.g.:

\begin{lstlisting}
perl valpsn.pl -ini=ini/val_ex_1.ini
\end{lstlisting}

\noindent The script will now run through the validation steps specified in the
configuration file, in each step first running the PsN command (if
requested), and next invoking the R script to compare the output with
the reference. At the end of the validation, \ValPsN will report a
summary of the validation procedure, e.g.:

\begin{lstlisting}
-------------------------------------------------------------------------------
Summary: Succes (10/10), failed (0/10), not tested (0/10)
-------------------------------------------------------------------------------
\end{lstlisting}

\noindent If tests have failed, it will report which tests have
failed, so they can be checked and run again separately.

\subsection{Getting help}
\noindent Similar to PsN toolkit commands, you can
run 
\begin{lstlisting}
perl valpsn.pl -h
\end{lstlisting}
or 
\begin{lstlisting}
perl valpsn.pl -help
\end{lstlisting} 
\noindent to get some more help on the available command line
arguments. Currently, the following arguments are available:

\begin{lstlisting}
  valpsn

    Running validation test on PsN toolkit commands.

        [ -h | -? ]
        [ --help ]
        [ --ini=<ini-file> ]
        [ --report=<report-file> ]
        [ --force_run_psn=0|1 ]
        [ --force_run_test=0|1 ]
        [ --force=0|1 ]
        [ --run_only=1,2,3,...]

\end{lstlisting}

\newpage

\section{Validation of PsN toolkit commands}

Below is a reference guide for valiation options split per PsN toolkit
command.

\vspace{10pt}

\noindent \textbf{Disclaimer}: The descriptions given for the PsN
tools are reproduced from the PsN website
(http://psn.sourceforge.net), if available. Also note that PsN is a
tool in constant development, and default PsN settings or algorithms may change
between versions. It is therefore advised to be as explicit as
possible in specifying the PsN command to run in the
validation procedure. Output format from PsN commands may also change between
versions, and the R scripts may therefore need some adaptation over
time.

\subsection{Tool-specific settings}
Settings specified in the tool-specific sections override those in the
{\tt [validation]} section in the header of the configuration
file. Tests implemented in the R script are only implemented if they
are set to 1 (or TRUE):

\begin{lstlisting}
test_ofv=1  # Compare ofv
\end{lstlisting}

If all available tests are to be implemented for a specific tool /
model, this can be achieved by specifying:

\begin{lstlisting}
test_all=1  # Compare all
\end{lstlisting}

\subsubsection{execute}
The \psn{execute} script is a PsN tool that allows you to run multiple
modelfiles either sequentially or in parallel. It is an nmfe
replacement with advanced extra functionality. \psn{Execute} creates
subdirectories where it puts NONMEMs input and output files, to make
sure that parallel NONMEM runs do not interfere with each other. The
top directory is by default named 'modelfit\_dirX' where 'X' is a
number that starts at 1 and is increased by one each time you run
\psn{execute}.

\vspace{10pt}

\noindent The \psn{execute} command performs a single NONMEM run. The
output from execute is non-numeric, i.e. it is a flat text file
(usually {\tt .lst} or {\tt .res} file)) that contains a lot of info
about the model, dataset and run process. It does include the
parameter estimates, and possibly the estimates of uncertainty in
parameter estimates. To convert these parameter estimates information
into a numeric format (csv), the separate PsN tool \psn{sumo} is used
after completion of the \psn{execute}. Therefore, any validation of
\psn{execute} is also a validation of \psn{sumo}.

\begin{lstlisting}
[execute]
folder=examples1	       # The folder with the model to run, must be a
			       # subfolder from "lib_folder"
model=run4.mod		       # NM model
reference=run4.lst             # The reference NM output file in the lib_folder.
			       # By default equal to NM model filename with
			       # the default NM output extension
nm_version=default	       # Overrides default NM version.
extra_files=		       # If extra files are reqd for the run, these [x]
			       # will be copied to the run folder
extra_arguments=               # Optional arguments to specify to PsN [x]
R_script=execute.R  	       # Validation script (R). Optional, default is:
			       # <psn_command>.R
			       # All R-scripts are in the /R folder.
command_explicit=              # Explicitly specify the whole command line, [x]
			       # which overrides all other settings.
ofv_abs_tol=3                  # Absolute tolerance allowed in OFV.
			       # By default, the relative default tolerance is used.
run_psn=0                      # Actually run the PsN command for this tool?
			       # if 0 or FALSE, the command wil not be run
			       # and the test (if implemented) will be performed
			       # using previously generated output (if avail.)
			       # By default 1
run_test=0		       # Actually perform the test using the R-script?
			       # By default 1
## possible tests
test_ofv=TRUE
test_parameters=TRUE
## /tests
\end{lstlisting}

\noindent Tip: not all of the above arguments need to be included in
the ini-file. A simpler setup could be:

\begin{lstlisting}
[execute]
folder=examples1
model=run4.mod
## tests
test_all=1
## /tests
\end{lstlisting}

\subsubsection{bootstrap}
\psn{Bootstrap} is a tool for calculating bias, standard errors and
confidence intervals of parameter estimates. It does so by generating
a set of new datasets by sampling individuals with replacement from
the original dataset, and fitting the model to each new dataset, see
Efron B, An Introduction to the Bootstrap, Chap. \& Hall, London UK,
1993. To compute standard errors for all parameters of a model using
the non-parametric bootstrap implemented here, roughly 200 model fits
are necessary. To assess 95\% confidence intervals approximatly 2000
will suffice.

\begin{lstlisting}
[bootstrap]
run_psn=0                 # run the bootstrap? If 0, bootstrap output should already be present
run_test=0                # perform the actual validation test in R?
folder=examples1          # folder containing the model to be used in the bootstrap
model=run4.mod            # model to use in the bootstrap
reference=bootstrap_dir1  # the bootstrap reference folder, in which a
			  # file called raw_results_<run>.csv and
			  # bootstrap_results.csv should be present.
seed=12345                # seed to be used for randomization
samples=100               # number of samples to use in the bootstrap
tolerance=0.01            # relative tolerance (for the parameters)
ofv_abs_tol=3             # absolute tolerance (for the OFV)

## tests
test_ofv_mean=1             # test on difference in mean OFV
test_ofv_sd=1               # test on difference in sd OFV
test_parameter_mean=1       # test on difference in mean of parameters
test_parameter_sd=1         # test on difference in sd of parameters
test_bootstrap_ofv=1	    # Compare OFVs for every
			    # bootstrap sample with the
		 	    # reference OFVs. Only makes sense if
		 	    # -seed is set!
test_bootstrap_parameters=0 # Compare parameters for every
			    # bootstrap sample with the
		 	    # reference. Only makes sense if
		            # -seed is set!
## /tests
\end{lstlisting}

\subsubsection{vpc / npc}
\psn{NPC} – Numerical Predictive Check – is a model diagnostics
tool. \psn{VPC} – Visual Predictive Check – is another closely related
diagnostics tool. A set of simulated datasets are generated using the
model to be evaluated. Afterwards the real data observations are
compared with the distribution of the simulated observations. By
default no model estimation is ever performed. The input to the
\psn{npc} script is the model to be evaluated, the number of samples
(simulated datasets) to generate, parameter values options for the
simulations, and stratification options for the evaluation. It is also
possible to skip the simulation step entirely by giving two already
created tablefiles as input. The input to the \psn{vpc} script is the
input to \psn{npc} plus an additional set of options.

\vspace{10pt}

\noindent Since a large portion of the processing is
identical between the scripts, and the numeric output for the
\psn{vpc} and \psn{npc} commands are the same, a separate validation
test for \psn{npc} is currently not available in ValPsN.

\begin{lstlisting}
[vpc]
model=run4.mod
folder=NM_examples
run_psn=0
run_test=1
seed=12345
samples=500
reference=vpc_dir1  # the vpc reference folder, in which a
	            # file called raw_results_<run>*.csv and the file
		    # vpc_results.csv should be present.
tolerance=0.05      # If seed is not used, tolerance should probably be fairly large
extra_arguments=-bin_by_count -no_of_bins=8 -dv=CP

##tests
test_obs_median=1   # tests median for the observed data
test_obs_5=1        # tests 5th pctile of the observed data
test_obs_95=1       # tests 95th pctile of the observed data
test_pi_median=1    # tests median of the simulated data
test_pi_median_ci=1 # tests the ci of the median of the simulated data
test_pi_95=1        # tests the 95th pctile of the simulated data
test_pi_5=1         # tests the 5th pctile of the simulated data
# or: test_all=1    # perform all of the above tests
## /tests
\end{lstlisting}

\subsubsection{cdd}
The Case Deletions Diagnostics (\psn{cdd}) algorithm is a tool
primarily used to identify influential components of the dataset,
usually individuals. The \psn{cdd} works by identifying groups in the
data set and creating one new data set for each member of the group,
where that member has been removed.  The model is run once with each
new data set. The PsN implementation of the \psn{cdd} can take any
column as base for the grouping and all rows with the same value in
that column will be considered a group as long as no individual
contain multiple values in that column. One should take care that the
grouping creates sensible individual records. PsN will renumber the ID
column so that two individuals with the same ID will not end up next
to each other.

\begin{lstlisting}
[cdd]
run_psn=0
run_test=1
folder=examples2
model=run4.mod
case_column=CENT
reference=cdd_dir1
ofv_abs_tol=1
## tests
test_jackknife_par_bias=1
test_jackknife_ofv_bias=1
## /tests
\end{lstlisting}

\subsubsection{llp}
The Log Likelihood Profiling (\psn{llp}) tool is used to calculate
confidence intervals of parameter values.  Without the \psn{llp} the
confidence intervals can be calculated with the standard errors of the
parameters under the assumption that the parameter values are normally
distributed. The \psn{llp}, however, makes no assumption of the shape
of the distribution.  The \psn{llp} tool will calculate the confidence
intervals for any number of parameters in the model, working with one
parameter at a time. By first fitting the original model and then
fixing the parameter at values close to the NONMEM estimate, the
\psn{llp} obtains the difference in likelihood between the original
model and new, reduced model. The logarithm of the difference in
likelihood is $\chi^2$ distributed and when that value is 3.84, the
parameter value is at the 95\% confidence limit. The search for the
limit is done on both sides of the original parameter value, and thus
the \psn{llp} makes no assumption of symmetry or the parameter value
distribution

\begin{lstlisting}
[llp]
folder=examples1
model=run4.mod
extra_arguments=-thetas='1,2' -rse_thetas='20,20'
reference=llp_dir1  # the llp reference folder, in which a
	            # file called raw_results_<run>*.csv and the file
		    # llp_results.csv should be present.
run_psn=0
run_test=1
## tests
test_ci=1           # compare confidence intervals
test_ofv=1          # compare OFVs
## /tests
\end{lstlisting}

\subsubsection{scm}
The Stepwise Covariate Model (\psn{scm}) building tool of PsN
implements Forward Selection and Backward Elimination of covariates to
a model. In short, one model for each relevant parametercovariate
relationship is prepared and tested in a univariate manner. In the
first step the model that gives the best fit of the data according to
some criteria is retained and taken forward to the next step. In the
following steps all remaining parameter-covariate combinations are
tested until no more covariates meet the criteria for being included
into the model. The Forward Selection can be followed by Backward
Elimination, which proceeds as the Forward Selection but reversely,
using stricter criteria for model improvement. 

\begin{lstlisting}
[scm]
run_psn=0
run_test=0
folder=PSP
model=run4.mod  # not required
extra_files=psp.scm
extra_arguments=-config_file=psp.scm
reference=scm_run5
## tests
test_final_model_same=1  # test if final covariate model is the same
test_ofv_final_model=1   # test if OFV for final covariate model is
the same
## /tests
\end{lstlisting}

\subsubsection{xv-scm}
Cross-validated scm, \psn{xv\_scm}, depends heavily on the \psn{scm} program, and
all \psn{scm} options apply also to \psn{xv\_scm} except that options
search\_direction, gof, p\_value, p\_forward, p\_backward and
update\_derivatives are ignored. A word of caution: xv\_scm produces
many files and takes up much disk space. 

\vspace{10pt}

\noindent \textit{Currently not covered in manual.}

\subsubsection{boot-scm}
Bootstrap scm, boot\_scm, depends heavily on the scm program, and all
scm options apply also to boot\_scm. Please refer to
scm\_userguide.pdf for help on scm options.

\vspace{10pt}

\noindent \textit{Currently not covered in manual.}

\subsubsection{sse}
SSE – Stochastic Simulation and Estimation – is a tool for model comparison and hypothesis testing.
First, using a given model, henceforth called the input model, a number of simulated datasets are
generated. Then the input model and a set of alternative models are fitted to the simulated data. Finally,
a set of statistical measures are computed for the parameter estimates and objective function values of
the various models.

\vspace{10pt}

\noindent \textit{Currently not covered in manual.}

\subsubsection{mcmp}
The Monte Carlo Mapped Power (MCMP) method provides a fast and
accurate prediction of the power and sample size
relationship. Efficient power calculation methods have been suggested
for Wald test-based inference in mixed-effects models but the only
available alternative for Likelihood ratio test-based hypothesis
testing has been to perform computer-intensive multiple simulations
and re-estimations. The MCMP method is based on the use of the
difference in individual objective function values ($\Delta$iOFV)
derived from a large dataset simulated from a full model and
subsequently re-estimated with the full and reduced models.

\vspace{10pt}

\noindent See Vong C. et al, AAPSJ 2012 Jun; 14(2):176-86.

\vspace{10pt}

\noindent \textit{Currently not covered in manual.}

\subsubsection{lasso}
Covariate models for population pharmacokinetics and pharmacodynamics
are often built with a stepwise covariate modelling procedure (SCM,
available in PsN). When analysing a small dataset this method may
produce a covariate model that suffers from selection bias and poor
predictive performance. The lasso is a method suggested to remedy
these problems. It may also be faster than SCM and provide a
validation of the covariate model. In the lasso all covariates must be
standardised to have zero mean and standard deviation
one. Subsequently, the model containing all potential covariate–
parameter relations is fitted with a restriction: the sum of the
absolute covariate coefficients must be smaller than a value, t. The
restriction will force some coefficients towards zero while the others
are estimated with shrinkage. This means in practice that when fitting
the model the covariate relations are tested for inclusion at the same
time as the included relations are estimated. For a given SCM
analysis, the model size depends on the P-value required for
selection. In the lasso the model size instead depends on the value of
\textit{t} which can be estimated using cross-validation.

\vspace{10pt}

\begin{lstlisting}
[lasso]
run_psn=0
run_test=1
folder=examples2
model=run4.mod
lst=run4.lst
relations=CL:WT-2,AGE-2,SEX-1,CENT-1,,V1:WT-2,AGE-2,SEX-1,CENT-1
reference=lasso_dir1
## tests
test_coef=1  # compare the coefficients for the covariate relationships
## /tests
\end{lstlisting}

\newpage

\section{Custom R scripts}
While default R scripts are provided that perform validations for
output from most PsN commands, custom R scripts can be used as well to
perform the validation. If additional tests are required, it is
probably easiest to use the existing default R-script as starting
point. Below is some guidance on how the R-scripts are implemented
within the \ValPsN framework.

\vspace{10pt}

\noindent First, in the configuration file for the validation, the custom R-script needs to
be specified:

\begin{lstlisting}
[execute]
model=run4.mod
reference=run4.lst
...
R_script=execute_custom.R
...
\end{lstlisting}

\noindent In which it is assumed that {\tt execute\_custom.R} is located within
the R-folder specified in the {[folder]} settings.

\vspace{10pt}

\noindent When the script is invoked, a copy of the script is made. To
this copy a header is prepended that specifies the current folder
and the validation settings, e.g. :

\begin{lstlisting}
cwd <- "/Users/username/ValPsn/runs/PsN_3.5.3_20120930/test1_execute"
setwd(cwd)
args <- list (
  command_explicit = "",
  extra_arguments = "",
  extra_files = "",
  folder = "examples1",
  model = "run4.mod",
  nm_mod_extension = "mod",
  nm_out_extension = "lst",
  nm_output = "execute_test_dir/NM_run1/psn.lst",
  nm_version = "default",
  ofv_abs_tol = "1",
  R_script = "execute.R",
  reference = "run4.lst",
  run_psn = "0",
  run_test = "1",
  test_ofv = 1,
  test_parameters = 1,
  threads = "5",
  tolerance = "0.01",
  verbose_level = "2"
)
\end{lstlisting}

\noindent So in the R-script below this header, the settings in the
{\tt args} list-object can be used to implement a validation
test. \ValPsN will also copy the required result files from the
reference output to the current folder (for most toolkit commands
that means the csv-files, but for some also log-files and raw results
files are copied). So if both the test and reference output are
imported into R, any test can be performed on these data. Please have
a look in the default R scripts how this is implemented. The
validation framework requires that at the end of the R-script, R needs
to report back whether the test succeeded or failed, which is
implemented e.g. as follows:

\begin{lstlisting}
## Write overall test succes
if (all_res) {
  cat ("Overall test result: SUCCESS\n")
} else {
  cat ("Overall test result: FAILED!\n")
}
\end{lstlisting}

\noindent in which {\tt all\_ref} is a boolean variable specifying whether the
test was successful or not. The syntax of the output should be
exactly as specified here to be interpreted by the framework.

\end{document}

